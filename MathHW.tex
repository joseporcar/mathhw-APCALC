% TODO
% an index that links to each hw

\documentclass[12pt]{article}
\usepackage{amsmath, multicol}
\usepackage{nopageno}
\usepackage{hyperref}
\usepackage{geometry}
\geometry{a4paper, bottom=10mm, textwidth=190mm}
\parindent 0in

\begin{document}
\title{\Huge \textbf {Math Homework}}
\author{José Porcar}
\date{Semester 2 2025}
\maketitle
\tableofcontents
\newpage

\section{Riemann and Trapezoidal sums}

\subsection*{Left Riemann Sums}
\begin{multicols}{2}

$f(x)=x+6; \; [1,5]$; 4 rectangles

\begin{center}
    \begin{tabular}{ c|c|c|c|c|c }
        $x$ & 1 & 2 & 3 & 4 & 5 \\
        \hline
        $y$ & 7 & 8 & 9 & 10 & 11 \\
    \end{tabular}
    \begin{align*}
        \triangle x &= \frac{|a-b|}{r} \\
        \triangle x &= \frac{|1-5|}{4} \\
        \triangle x &= 1 \\
        A &= \sum_{n=a}^{b-1} f(n)\triangle x \\ 
        A &= \sum_{n=1}^4 f(n) \\
        A &= 7 + 8 + 9 + 10 \\
        A &= 34
    \end{align*}
\end{center}

\columnbreak 
$f(x)=x+4; \; [-2,2]$; 4 rectangles

\begin{center}
    \begin{tabular}{ c|c|c|c|c|c }
        $x$ & -2 & -1 & 0 & 1 & 2 \\
        \hline
        $y$ & 2 & 3 & 4 & 5 & 6 \\
    \end{tabular}
    \begin{align*}
        \triangle x &= \frac{|a-b|}{r} \\
        \triangle x &= \frac{|-2-2|}{4} \\
        \triangle x &= 1 \\
        A &= \sum_{n=a}^{b-1} f(n)\triangle x \\ 
        A &= \sum_{n=-2}^1 f(n) \\
        A &= 2 + 3 + 4 + 5 \\
        A &= 14
    \end{align*}
\end{center}
\end{multicols}

\subsection*{Right Riemann Sums}
\begin{multicols}{2}

$f(x)=-x^2-2x+9; \; [-3,2]$; 5 rectangles

\begin{center}
    \begin{tabular}{ c|c|c|c|c|c|c }
        $x$ & -3 & -2 & -1 & 0 & 1 & 2 \\
        \hline
        $y$ & 6 & 9 & 10 & 9 & 6 & 1\\
    \end{tabular}
    \begin{align*}
        \triangle x &= \frac{|a-b|}{r} \\
        \triangle x &= \frac{|-3-2|}{5} \\
        \triangle x &= 1 \\
        A &= \sum_{n=a+1}^{b} f(n)\triangle x \\ 
        A &= \sum_{n=-2}^2 f(n) \\
        A &= 9 + 10 + 9 + 6 + 1 \\
        A &= 35
    \end{align*}
\end{center}

\columnbreak 
$f(x)=\frac{2}{x}; \; [2,7]$; 5 rectangles

\begin{center}
    \begin{tabular}{ c|c|c|c|c|c|c }
        $x$ & 2 & 3 & 4 & 5 & 6 & 7 \\
        \hline
        $y$ & 1 & $\frac{2}{3}$ & $\frac{1}{2}$ & $\frac{2}{5}$ & $\frac{1}{3}$ & $\frac{2}{7}$ \\
    \end{tabular}
    \begin{align*}
        \triangle x &= \frac{|a-b|}{r} \\
        \triangle x &= \frac{|2-7|}{5} \\
        \triangle x &= 1 \\
        A &= \sum_{n=a+1}^{b} f(n)\triangle x \\ 
        A &= \sum_{n=3}^7 f(n) \\
        A &= \frac{2}{3} + \frac{1}{2} + \frac{2}{5} + \frac{1}{3} + \frac{2}{7} \\
        A &= 2.19
    \end{align*}
\end{center}

\end{multicols}
\newpage
\subsection*{Trapezoidal Sums}
\begin{multicols}{2}

$y = -\frac{x^2}{2} + x + 5; \; [0, 3]$; 5 Trapezoids

\begin{center}
    \begin{tabular}{ c|c|c|c|c|c|c }
        $x$ & 0 & 0.6 & 1.2 & 1.8 & 2.4 & 3 \\
        \hline
        $y$ & 5 & 5.42 & 5.48 & 5.18 & 4.52 & 3.5\\
    \end{tabular}
    \begin{align*}
        \triangle x &= \frac{|a-b|}{r} \\
        \triangle x &= \frac{|0-3|}{5} \\
        \triangle x &= 0.6 \\
        A &= \sum_{n=1}^{rectangles} \frac{f(n\triangle x)+f((n-1)\triangle x)}{2}\triangle x \\ 
        A &= \sum_{n=1}^{5} \frac{f(0.6n)+f(0.6(n-1))}{2} 0.6 \\ 
        A &= 3.13 + 3.27 + 3.23 + 2.91 + 2.41 \\
        A &= 14.95
    \end{align*}
\end{center}

\columnbreak 
$f(x)=\frac{x^2}{2} + x + 1; \; [-2,1]$; 5 Trapezoids

\begin{center}
    \begin{tabular}{ c|c|c|c|c|c|c }
        $x$ & -2 & -1.4 & -0.8 & -0.2 & 0.4 & 1 \\
        \hline
        $y$ & 1 & 0.58 & 0.52 & 0,82 & 1,48 & 2,5 \\
    \end{tabular}
    \begin{align*}
        \triangle x &= \frac{|a-b|}{r} \\
        \triangle x &= \frac{|-2-1|}{5} \\
        \triangle x &= 0.6 \\
        A &= \sum_{n=1}^{rectangles} \frac{f(n\triangle x)+f((n-1)\triangle x)}{2}\triangle x \\ 
        A &= \sum_{n=1}^{5} \frac{f(0.6n)+f(0.6(n-1))}{2} 0.6 \\ 
        A &= 0,474 + 0.33 + 0.402 + 0.69 + 1.194 \\
        A &= 3.09
    \end{align*}
\end{center}

\end{multicols}
\newpage
\section{Volume of Solids of Revolutions (Disk Method)}
\begin{enumerate}
    \item[3.] Find the volume of the solid of revolution generated by revolving the region bounded by $y=\sqrt{1-x^2}$ and $y=0$ about the $x$-axis.
    \begin{align*}
        V&=\pi\int^a_b\left[r^2\right]dx \\
         &=\pi\int^1_{-1}\left[\left(\sqrt{1-x^2}\right)^2\right]dx \\
         &=\pi\int^1_{-1}\left[1-x^2\right]dx \\
         &=\pi\left[x-\frac{x^3}{3}\right]^1_{-1} \\
         &=\pi\left(1-\frac{1}{3}-\left(-1-\frac{-1}{3}\right)\right) \\
        V&=\frac{4\pi}{3}
    \end{align*}
    \item[8.]Find the volume of the solid formed by revolving the region bounded by the graphs of \( y = \sqrt{\sin x} \) and the \( x \)-axis (\( 0 \leq x \leq \pi \)), about the \( x \)-axis.
    \begin{align*}
        V&= \pi \int_{0}^{\pi} (\sqrt{\sin x})^2 \, dx \\
         &= \pi \int_{0}^{\pi} \sin x \, dx \\
         &= \pi \left[ -\cos x \right]_{0}^{\pi} \\
         &= \pi \left( -(-1) + 1 \right) \\
        V&=2\pi \\
    \end{align*}
    \item[11.] The region bounded by the parabola \( y = 4x - x^2 \) and the \( x \)-axis is revolved about the \( x \)-axis. Find the volume of the solid.
    \begin{align*}
        \{a,b\}&=x \text{\quad where }4x-x^2=0 \\
        a=0 ;&\quad b=4 \\
        V &= \pi \int_{0}^{4} (4x - x^2)^2 \, dx \\
        &=\pi \int_{0}^{4} (16x^2 - 8x^3 + x^4) \, dx \\
        &=\pi \left[ \frac{16}{3}x^3 - 2x^4 + \frac{1}{5}x^5 \right]_{0}^{4} \\
        &=\pi\left(\left(\frac{1024}{3}-512+\frac{1024}{5}\right)-0\right)\\
        V&=\frac{512\pi}{15}
    \end{align*}
    \item[13a.]Find the volume of the solid generated when the area bounded by the curves \( y = x^3 - x + 1 \), \( x = -1 \), and \( x = 1 \) is revolved around the \( x \)-axis.
    \begin{align*}
        V &= \pi \int_{-1}^{1} (x^3 - x + 1)^2 \, dx \\
        &= \pi \int_{-1}^{1} (x^6 - 2x^4 + 2x^3 + x^2 - 2x + 1) \, dx \\
        &= \pi \left[ \frac{1}{7}x^7 - \frac{2}{5}x^5 + \frac{1}{2}x^4 + \frac{1}{3}x^3 - x^2 + x \right]_{-1}^{1} \\
        &= \pi \left( \left( \frac{1}{7} - \frac{2}{5} + \frac{1}{2} + \frac{1}{3} \right) - \left( -\frac{1}{7} + \frac{2}{5} + \frac{1}{2} - \frac{1}{3} - 2 \right) \right) \\
       V &=\frac{226\pi}{105} \text{ source: deepseek I aint doing them fractions :b}
    \end{align*}
    \item[13b.]Find the volume of the solid generated when the area bounded by the curve \( y = x - x^2 \) and the \( x \)-axis is revolved around the \( x \)-axis.
    \begin{align*}
        V&= \pi \int_{0}^{1} (x - x^2)^2 \, dx \\
        &= \pi \int_{0}^{1} (x^2 - 2x^3 + x^4) \, dx \\
        &= \pi \left[ \frac{1}{3}x^3 - \frac{1}{2}x^4 + \frac{1}{5}x^5 \right]_{0}^{1} \\
        &= \pi \left( \frac{1}{3} - \frac{1}{2} + \frac{1}{5} \right) \\
        V&=\frac{\pi}{30}
    \end{align*}
\end{enumerate}

\newpage
\section{Volume of Solids of Revolutions (Washer Method)}
Find the volume of the solid that results when the region enclosed by the curves is revolved about the given axis.
\begin{multicols}{2}
\begin{center}
    $f(x)=x^2-3,\quad g(x)=\sqrt{x}-3;\quad$axis:$\;y=2$
    \begin{align*}
        f(x) &= g(x) \\
        x^2-3&=\sqrt{x}-3 \\
        x^2&=\sqrt{x}\\
        x^4&=x\\
        x\left(x^3-1\right)&=0\\
        x=0\ &\qquad\! x^3=1 \\
        a=0\ &\qquad\! b=1 \\
        V=\pi\int^b_a\Bigl|(f(x)-y)^2 &-(g(x)-y)^2\Bigr|\;dx \\
        =\pi\int^1_0\Bigl|(x^2-5)^2 &-(\sqrt{x}-5)^2\Bigr|\;dx \\
        =\pi\int^1_0\Bigl|x^4-10x^2+25 &-(x-10\sqrt{x}+25)\Bigr|\;dx \\
        =\pi\Biggl|\frac{x^5}{5}-\frac{10x^3}{3} &- \frac{x^2}{2}+\frac{20x^{\frac{3}{2}}}{3}\Biggr|^1_0 \\
        =\pi\biggl|\frac{1}{5}-\frac{10}{3}&-\frac{1}{2}+\frac{20}{3}\biggr| \\
        V&=\frac{91\pi}{30} \\
    \end{align*} 
    \vspace{3pt}\\
    $f(x)=x^2+4,\quad g(x)=2;\;a=0,\;b=1;$\\
    \raggedleft axis:$\;y=-1\quad$ 
    \begin{align*}
        V=\pi\int^b_a\Bigl|(f(x)-y)^2 &-(g(x)-y)^2\Bigr|\;dx \\
        =\pi\int^1_0\Bigl|(x^2&+5)^2 -3^2\Bigr|\;dx \\
        =\pi\int^1_0\Bigl|x^4+&10x^2+16\Bigr|\;dx \\
        =\pi\Biggl|\frac{x^5}{5}+&\frac{10x^3}{3} + 16x\Biggr|^1_0 \\
        =\pi\biggl|\frac{1}{5}+&\frac{10}{3}+16\biggr| \\
        V&=\frac{293\pi}{15} \\ 
    \end{align*} 
    
\columnbreak
    $f(x)=x^2-3,\quad g(x)=\sqrt{x}-3;\quad$ axis:$\;y=1$
    \begin{align*}
        f(x) &= g(x) \\
        x^2-3&=\sqrt{x}-3 \\
        x^2&=\sqrt{x}\\
        x^4&=x\\
        x\left(x^3-1\right)&=0\\
        x=0\ &\qquad\! x^3=1 \\
        a=0\ &\qquad\! b=1 \\
        V=\pi\int^b_a\Bigl|(f(x)-y)^2 &-(g(x)-y)^2\Bigr|\;dx \\
        =\pi\int^1_0\Bigl|(x^2-4)^2 &-(\sqrt{x}-4)^2\Bigr|\;dx \\
        =\pi\int^1_0\Bigl|x^4-8x^2+16 &-(x-8\sqrt{x}+16)\Bigr|\;dx \\
        =\pi\Biggl|\frac{x^5}{5}-\frac{8x^3}{3} &- \frac{x^2}{2}+\frac{16x^{\frac{3}{2}}}{3}\Biggr|^1_0 \\
        =\pi\biggl|\frac{1}{5}-\frac{8}{3}&-\frac{1}{2}+\frac{16}{3}\biggr| \\
        V&=\frac{71\pi}{30} \\
    \end{align*} 
    \vspace{3pt}\\
    $f(x)=x^2+3,\quad g(x)=3;\;a=0,\;b=2;$\\
    \raggedleft axis:$\;y=1\quad$ 
    \begin{align*}
        V=\pi\int^b_a\Bigl|(f(x)-y)^2 &-(g(x)-y)^2\Bigr|\;dx \\
        =\pi\int^2_0\Bigl|(x^2&+2)^2 -2^2\Bigr|\;dx \\
        =\pi\int^2_0\Bigl|x^4&+4x^2\Bigr|\;dx \\
        =\pi\Biggl|\frac{x^5}{5}&+\frac{4x^3}{3} \Biggr|^2_0 \\
        =\pi\biggl|\frac{32}{5}&+\frac{32}{3}\biggr| \\
        V&=\frac{256\pi}{15} \\ 
    \end{align*} 
\end{center}
\end{multicols}
\end{document}